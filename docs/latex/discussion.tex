\section{Discussion}
\label{sec:discussion}
Using static analysis to develop system wide policies has its limitations. The correctness of the policies depends upon the code coverage of the system. While DockerGate does analyze most of the ELF executables and shared objects, executable code like Java JAR files still remains out of scope for DockerGate. Since many images do use Java, this poses to be a threat to validity as the system calls these files make are left out of the policy.

In our ELF shared-object analysis, we have tracked the contents of the EAX register till just before the system call. However, while we do handle the "mov" instruction recursively as explained in Section~\ref{sec:design}, there were some cases where the contents of EAX were being modified by other instructions such as XOR and ADD. Such examples were fewer in nature. But we had to log them as exceptions that DockerGate could not handle.

We have focused solely on static analysis in DockerGate because of the reasons cited in 
Section~\ref{sec:background}. We believe that if a technique could be found that could execute all possible branches of the executable code in a Docker container, dynamic analysis could be combined with DockerGate to provide tighter policies. Such work has been done previously in Android Applications~\cite{monkeyrunner} and could be expanded to Docker containers. While static analysis gives a complete overview of what system calls are required, the dynamic analysis could remove the system calls present in dead-code or code that can never be reached during the execution of the application in the Docker container.

DockerGate can also be extended to produce AppArmour~\cite{apparmor} profiles. AppArmour profiles can control the permissions the program in a container can have. These permissions range from read/write/execute abilities on certain files to network access. AppArmour and Seccomp profiles combined can further reduce the attack surface on the host kernel as an attacker would not be able to use an existing program outside the bounds defined by both the security modules.
