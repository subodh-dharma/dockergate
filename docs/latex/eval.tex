\section{Evaluation}
\label{sec:eval}

\subsection{Characterizing Data Set}

Docker Hub currently host around 400,000 images. To perform experiments with DockerGate, it was required to limit our dataset to test and evaluate. Docker Hub hosts two types of images viz. Official images and Community Images. Official images are maintained by official owners of applications and community images are maintained by users and community members which contain some customized applications developed on top of official images.

To characterize the images we used the Dockerfiles of each image. Dockerfile for Official Images were obtained from the links consolidated in Docker’s Official Library repository[9]. While Dockerfile(s) for community images were obtained using a web scraper. We used ScraPy~\cite{scrapy}, a web scraping framework, to scrape Docker Hub for image names. Using random English dictionary words as search keywords, we were able to get about 96,000 Docker image names. Because of limitations in resources, we randomly sampled 110 Docker image names from those 96,000 names. We conduct our further experiments on this sample of 110 Docker images.

We analyzed these Dockerfiles to determine the nature of docker images. We first identified the base image of the docker image, which can be obtained from Dockerfile’s first line starting with keyword FROM. We analyzed the official images and community images separately. The result of this analysis can be found in Table~\ref{topbaseimages}.

As can be observed from Table~\ref{topbaseimages}, out of total 536 Dockerfiles, 118 images have Openjdk base images. Since Docker images can be derived from each other, we investigated the base images for OpenJDK and buildpack-deps. It was found that all these images were based on either Ubuntu, Alpine or Debian images as their most lowest level base image. The majority of the images were based on Ubuntu and Debian. A complete list of the distribution of images can found in the Appendix \ref{sec:appendix}.

Further we analyzed Dockerfiles of the community images obtained using web scraper. We randomly selected 400 images from the set and checked for the base image. It was observed that about 250 images had a base image of Ubuntu and Debian. While  about 100 images were based on Alpine, the rest of them were distributed equally among Fedora, CentOS and Scratch(Not based on any prior Docker Image).


Since the Ubuntu and Debian images was more popular in usage for developing the custom docker images, we have considered only Ubuntu or Debian based images for the analysis and testing of DockerGate.

\begin{table}
\centering
\resizebox{\columnwidth}{!}{%
\begin{tabular}{|l|l|}
\hline
\textbf{Base Image} & \textbf{Number of docker images} \\ \hline
openjdk & \multicolumn{1}{c|}{118} \\ \hline
buildpack-deps & \multicolumn{1}{c|}{79} \\ \hline
debian & \multicolumn{1}{c|}{78} \\ \hline
alpine & \multicolumn{1}{c|}{73} \\ \hline
scratch & \multicolumn{1}{c|}{49} \\ \hline
php & \multicolumn{1}{c|}{42} \\ \hline
python & \multicolumn{1}{c|}{15} \\ \hline
microsoft/windowsservercore & \multicolumn{1}{c|}{12} \\ \hline
microsoft/nanoserver & \multicolumn{1}{c|}{9} \\ \hline
ubuntu & \multicolumn{1}{c|}{9} \\ \hline
\end{tabular}
}
\caption{Top 10 base images used in building a Docker Image}
\label{topbaseimages}
\end{table}
We then analyzed the average file composition of a Docker Image. Out of a random sample of 110 Docker images, we analyzed the file types for all the executable code in each. As shown in Table ~\ref{filecompose}, it was found that ELF files and Python scripts made up most of the executable code in an average Docker Image. So, we decided to analyze ELF files to generate the Seccomp Policies. This was also supported by the fact that Python files were linked to ELF shared-object libraries. So by analyzing ELF files, we could provide some coverage for Python executable code as well.

\begin{table}
\centering
\resizebox{\columnwidth}{!}{%
\begin{tabular}{|l|l|}
\hline
\textbf{Container Status} & \textbf{Number of docker images} \\ \hline
Successful Execution & \multicolumn{1}{c|}{80} \\ \hline
Failed Execution & \multicolumn{1}{c|}{30} \\ \hline
\textbf{Total} & \multicolumn{1}{c|}{\textbf{110}} \\ \hline
\end{tabular}
}
\caption{Container status after loading them with DockerGate generated Seccomp policy}
\label{tab:seccompresults}
\end{table}

\begin{table}[b]
\centering
\begin{tabular}{|l|c|}
\hline
\textbf{File Type} & \textbf{No. of Files (in percentage)} \\ \hline
ELF Files          & 28                                    \\ \hline
Python Scripts     & 35                                    \\ \hline
Shell Scripts      & 14                                    \\ \hline
Others             & 23                                    \\ \hline
\end{tabular}
\caption{Average File Composition for Executable code of a Docker Image}
\label{filecompose}
\end{table}



%\begin{figure}
%    \begin{tikzpicture}
%        \begin{axis}[
%            symbolic x coords={ELF, Python, Shell, Other},
%            xtick=data
%          ]
%            \addplot[ybar,fill=blue] coordinates {
%                (ELF,   28)
%                (Python,  35)
%                (Shell,   14)
%                (Other,   1.1)
%            };
%        \end{axis}
%    \end{tikzpicture}
%    \caption{Average Composition for Executable code of a Docker Image}
%    \label{filecompose}
%\end{figure}


\subsection{DockerGate Policy Evaluation}

%We use DockerGate by providing the name of a docker image as input to DockerGate, further it performs the analysis and updates the global database of system calls if required and generates a seccomp  policy. 

% Write Table Legend Here
% Replace following terms as in tables

After issuing a docker run command, the launched container can be in three possible states. 
\begin{itemize}
\item \textbf{Up} - the container is up and live, applications within the container are being executed as expected.
\item \textbf{Created} - in this state it creates a writable container layer but doesn’t start the container execution.
\item \textbf{Exit} - the container is destroyed and all the resources are freed. Depending on the exit code, it determines if container was gracefully destroyed or with some error in the application. Exit code of 0 indicates a graceful exit, while any other code indicates an erroneous exit from container
\end{itemize}

We determine the generated Seccomp policy for docker images to be successful by applying the generated policy and attempting to create and execute a simple container. If the status of the container is Up or if it exits with exit code 0, we assume that the container was created successfully and was able to execute the entry-point application within the container. We also verified the success of a container by executing the basic commands within the container. If the result of these commands was successful then we considered the container might function normally. The commands that we tested were \textit{uname, mkdir, /bin/sh \& echo}. If the container exited with a non-zero code or entry point failed to execute or appeared to hang-up, we considered it a failure in the Seccomp policy.
% Replace this with actual images results
From a set of Ubuntu based docker images, we executed DockerGate on randomly selected 110 community images. The images contained applications ranging from simple binary files to complete Databases and Web Servers. 





From the Table \ref{tab:seccompresults} we can see that 80 out of 110 containers exited in successful state. While 30 of them ended in failure. Further we analyzed the failed images, in which we observed 17 images crashed during execution i.e. they exited the container with status greater than 0 \{e.g. 1, 139, 159\}. The remaining 13 images out of 30 failed images ended up in \textbf{Created} state. The Created state indicates the seccomp policy was successful in initiating the containers but not successful in execution. The reason  we speculate is that the container didn't had any executable binaries but relied on script based languages or other native languages like JavaScript, Python or Java.   

The Failed images also included 3 images which had failed to execute because the seccomp policy generated didn't include some of the system calls. This caused the container to exit erroneously.

To determine which system calls were used during the execution of the container we used a tool called SystemTap~\cite{systemtap}. We developed a custom SystemTap script which would print the system calls being called during the execution of the container. The reason why we could trust System Tap in this case, as compared to strace is that, when a system tap script is executed, a custom Linux kernel module is created and inserted into the kernel. Based on the probe events defined in the script, when any such event is triggered corresponding action is executed in the kernel space. This allowed us to identify exactly which system calls had been used by any command, which in this case, was the Docker container. 

We spawned the Docker container for same 110 images used above and obtained the system calls called during the process. In the 110 images above, we found that an average of 118 system calls are actually required for spawning and initiating the start up process within the container.

Thus from the above experiments conducted, we can derive a lower bound on the number of system calls a docker image with base Ubuntu image can have. With an average of \textbf{minimum 118 system calls} a basic Ubuntu image can be  successfully executed, while our \textbf{DockerGate framework generates seccomp policy with average number of system calls equal to 213}. 

The current default seccomp policy includes about over 300 system calls to be allowed to be executed, which exposes a large attack surface for the host kernel. Based on the result of experiments conducted, the default policy can be reduced to 213 system calls identified above and reduce the attack surface of the host kernel. 

During the evaluation of the images we observed that 17 system calls were required to create a container from its image. Hence, these system calls were included by default when generating the policies. A full list of these system calls can be found in Appendix.