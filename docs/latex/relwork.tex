\section{Related Work}
\label{sec:relwork}

After the introduction of the containers, many developers have focused on container hardening which uses the kernel capabilities and features to protect and secure the containers. Various approaches have been proposed earlier where different kernel features can be used to secure the container. One such approach, as proposed by Rastogi et al.~\cite{rastogi}, is to divide a container into simpler containers to have limited functionality with least privilege, while preserving the overall functionality of the Image.  

Provos ~\cite{provos2003improving} proposed \textit{Systrace} to facilitate process-specific policy generation based on how the processes invoked system calls. This was a Dynamic Analysis technique that monitored the behaviour of the processes. Policy generation has also been attempted with SELinux ~\cite{sniffen2006guided, harada2003access} and with Android Applications ~\cite{afonso2016going}. However, with Docker being a young technology, it hasn't gained much traction on research in academia. But there are many community developers who attempt to bring tools or solutions in various aspects of the Docker. Docker-Slim~\cite{dockerslim}, is a similar project which primarily depends on dynamic analysis of the Docker containers. However, as discussed earlier in this paper, it is difficult to perform dynamic analysis because of the varied number of applications running inside the container. On the contrary, the static analysis scans all eligible files and  identify all possible system calls made by the binaries executables present in the docker image. As seen in ~\cite{afonso2016going}, the code coverage by dynamic analysis despite having an automation framework ranges from 8-10\%.  

One more kernel feature that can be exploited for container hardening is AppArmor. LiCShield~\cite{mattetti} exploits this feature by generating rules for restricting the access of a docker container by monitoring the changes made by processes associated with container and convert it into Linux security module for AppArmor. The works of LiCShield was based on Docker 1.6 version which didn’t had the support of Seccomp and AppArmor profiles, but were required to depend on host operating system to implement AppArmor. LiCShield was required to be supported by the host operating system and Docker was not involved in any way. They monitored and controlled the processes forked by Docker daemon.


There have been proposals~\cite{dockerpolicymodules} related to shipping a SELinux policy along with the docker image, where SELinux features are enabled on the host machine, and the SELinux policy is applied to the docker containers executed from this image. But this proposal requires that all the host operating systems must have SElinux policy installed and enabled which is by default is disabled for most Linux based systems. According to the study by Raj et al~\cite{raj2016}, using the security features like AppArmor and SELinux policies built in Docker, the various stages of DevOps pipeline can be secured and based on suitable profiles can secure the stages and deployments. Their experiments claim to mitigate certain security risk vectors.

Rajagopalan et al~\cite{rajgopalan} proposed Authenticated System Calls which performed static analysis on executable binaries to replace the system calls with custom system calls. But this required modification in the host kernel which may affect the integrity of host kernel.