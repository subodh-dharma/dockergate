\section{Introduction}

Linux Containers have been gaining attention in production cloud environments in recent times. There have been a wide range of products that are leveraging the container technology (LXC~\cite{lxc1}, OpenVZ~\cite{openvz}, Docker\cite{dockerarticle} etc.). Linux Containers are lightweight virtual machine processes which share the common host kernel and isolate the data and process from host using kernel security features like namespaces, cgroups and mandatory access controls~\cite{hardenlinuxcontainer}. The ability to virtualize an operating system into a single process without any additional virtual kernel makes the containers speed up the boot process~\cite{containerperformance} and makes the application easily available.

Docker~\cite{docker} is one of the technologies that leverages the use of Linux containers. The containers are based on LXC\cite{lxc1} and provide a virtualization framework for software level as well as hardware level~\cite{dockersaas}. By encapsulating all system dependencies in a single read-only file, Docker has made it easy to maintain and deploy an application on a large scale. There exist ready-to-use images~\cite{dockerimages} published by various vendors and community on Dockerhub~\cite{dockerhub}.The availability of Docker framework for different operating systems as well as different levels of infrastructure ranging from desktop to cloud service providers makes it more popular, easy and flexible to install and use. Based on a survey conducted in April 2017, around 12,947 companies are currently using Docker~\cite{survey1}. Moreover, there is research being done to introduce Docker to High Performance Computing~\cite{dockerHPC}. Docker has also been considered as a solution to the problem of not having access to reproducible research, especially in Computer Systems Research~\cite{collberg2014,dockerrepresearch2014}.

One of the major reasons for using Docker containers is the security they are able to provide. Since, all the processes and data within a container are inaccessible directly, it reduces the attack surface to the bare minimum of a single container. Common attacks like escalation-of-privilege attacks~\cite{jessehertz2016} can be avoided as the data and processes are executing in an isolated environment with no access to host resources. The Docker Security open-source community~\cite{dockeropensource} is also active in mitigating new threats on a regular basis. As of April 2017, there have been only 14 CVEs registered since 2014~\cite{cvelist}. The attack surface of docker is mainly on runC process, docker daemon and containerd process which are the building blocks of Docker containers 
% <probably expand more on some CVEs and explain if seccomp policies can reduce any of the CVE>

Docker containers use runC~\cite{solomonhykes2015} to start a container, which takes care of managing all the components required for a container which consist of  cgroups, namespaces and capabilities~\cite{dockersec1}. In the most recent release of docker v1.17, docker introduced a feature to support Seccomp~\cite{seccomp} profiles. The Seccomp profile allows restricting the usage of system calls made by processes within a container to its host kernel. The Seccomp profile is a simple file which can specify the action on a system call, whether to allow it or deny. When a container is launched with a Seccomp profile applied to it, it limits the number of system calls the container can make. If no explicit profiles are used for the container, a default profile is loaded implicitly which blocks access to 44 system calls and allows over 300 system calls~\cite{seccomp}. This default policy was created to be all-inclusive of every Docker Image so that their functionality is not hampered by Seccomp.

Many cloud hosting services like AWS, Digital Ocean etc. allow deployment and hosting of docker containers and related technologies~\cite{awsdigitalocean}. Deployment of a Docker container allows it to use most host operating system resources, which mainly includes system calls, files and access to network. Users can deploy and run containers with different types of applications and services running inside it. Currently there exist no measures for the host to check which processes are being executed inside the containers. While being a security feature, this isolation poses a certain risk to the host. The hosting vendors have to trust the Docker Images and containers to keep the Kernel safe from any possible attack. If the docker containers are executed with an image-specific Seccomp policy, then the possibility of any process using any vulnerable system calls that the application is not using, will be blocked. This will help the host vendors keep the host operating system safe from being exploited by container operations. For example, one would not be able to execute System Call Fuzzers ~\cite{kernelsyscallfuzzer} from inside a container.

To generate image-specific Seccomp policies and protect the host, we propose \textbf{DockerGate}, a framework which can be used to generate a custom Seccomp profile for a Docker image. The Seccomp profile generated should contain the system calls which must be allowed to be accessed by the container processes. With the generation of an image-specific Seccomp profile, we hypothesize that limiting the number of system calls accessible to a container, we can reduce the attack surface on the host kernel without affecting the functionality of the containerized application processes. We evaluate DockerGate by running the framework on a random sample of 200 Docker images and testing the successful start-up and functioning of each container. Each policy produced allows significantly less system calls than the system calls allowed by the default Seccomp policy provided by Docker.
\hfill \break
\\This paper makes the following contributions:
\begin{itemize}
\item
\textit{We propose DockerGate, an automated Seccomp policy generator for Docker Images.} Our approach involves statically analyzing all executable code in the Docker image and mapping the system calls that might be invoked from that code. Those system calls are aggregated to create a least-privileged Seccomp policy 
\end{itemize}
\begin{itemize}
\item 
\textit{We implement a proof-of-concept prototype for DockerGate that can be used to analyze Docker images based on Ubuntu~\cite{ubuntu}}. By using various Linux-based Binary Analysis tools~\cite{nm,ldd,objdump}, we were able to profile the ELF binaries in each image and relate them to the libraries they are dynamically linked to. We then mapped those invoked library functions to the system calls each function required and generated the Seccomp profile.
\end{itemize}
\begin{itemize}
\item
\textit{We evaluate DockerGate on a random sample of 20 Docker images}. We used DockerGate to generate a Seccomp policy for each image. We then tested each policy by applying them to a container running the specific image that policy was generated for.
\end{itemize}

 DockerGate produces a smaller, lesser privileged, image-specific Seccomp policy for each Docker image. This is able to reduce the attack surface for the host operating system while keeping the Docker container functioning properly.

The remainder of this paper proceeds as follows.
Section~\ref{sec:background} gives the background and motivation for the paper.
Section~\ref{sec:overview} gives an overview of the design for DockerGate.
Section~\ref{sec:design} gives a detailed description of the Design for DockerGate
Section~\ref{sec:eval} evaluates our solution.
Section~\ref{sec:discussion} discusses additional topics.
Section~\ref{sec:relwork} describes related work. Section~\ref{sec:conc}
concludes.
